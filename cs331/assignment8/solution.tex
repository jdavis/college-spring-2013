\documentclass{article}

\usepackage{fancyhdr}
\usepackage{lastpage}
\usepackage{extramarks}
\usepackage[usenames,dvipsnames]{color}
\usepackage{amsmath}
\usepackage{amsthm}
\usepackage{amsfonts}

\usepackage{tikz}

\topmargin=-0.45in
\evensidemargin=0in
\oddsidemargin=0in
\textwidth=6.5in
\textheight=9.0in
\headsep=0.25in

\linespread{1.1}

\pagestyle{fancy}
\lhead{\hmwkAuthorName}
\chead{\hmwkClass\ (\hmwkClassInstructor\ \hmwkClassTime): \hmwkTitle}
\rhead{\firstxmark}
\lfoot{\lastxmark}
\cfoot{}
\renewcommand\headrulewidth{0.4pt}
\renewcommand\footrulewidth{0.4pt}

\setlength\parindent{0pt}

\newcommand{\enterProblemHeader}[1]{
    \nobreak\extramarks{#1}{#1 continued on next page\ldots}\nobreak
    \nobreak\extramarks{#1 (continued)}{#1 continued on next page\ldots}\nobreak
}

\newcommand{\exitProblemHeader}[1]{
    \nobreak\extramarks{#1 (continued)}{#1 continued on next page\ldots}\nobreak
    \nobreak\extramarks{#1}{}\nobreak
}

\setcounter{secnumdepth}{0}
\newcounter{homeworkProblemCounter}

\newcommand{\homeworkProblemName}{}
\newenvironment{homeworkProblem}[1][Problem \arabic{homeworkProblemCounter}]{
    \stepcounter{homeworkProblemCounter}
    \renewcommand{\homeworkProblemName}{#1}
    \section{\homeworkProblemName}
    \enterProblemHeader{\homeworkProblemName}
}{
    \exitProblemHeader{\homeworkProblemName}
}

\newcommand{\problemAnswer}[1]{
\noindent\framebox[\columnwidth][c]{\begin{minipage}{0.98\columnwidth}#1\end{minipage}}
}

\newcommand{\homeworkSectionName}{}
\newenvironment{homeworkSection}[1]{
    \renewcommand{\homeworkSectionName}{#1}
    \subsection{\homeworkSectionName}
    \enterProblemHeader{\homeworkProblemName\ [\homeworkSectionName]}
}{
    \enterProblemHeader{\homeworkProblemName}
}

\newcommand{\hmwkTitle}{Homework\ \#8}
\newcommand{\hmwkDueDate}{April 11, 2013 at 11:59pm}
\newcommand{\hmwkClass}{CS331}
\newcommand{\hmwkClassTime}{9:00am}
\newcommand{\hmwkClassInstructor}{Professor Zhang}
\newcommand{\hmwkAuthorName}{Josh Davis}

\title{
    \vspace{2in}
    \textmd{\textbf{\hmwkClass:\ \hmwkTitle}}\\
    \normalsize\vspace{0.1in}\small{Due\ on\ \hmwkDueDate}\\
    \vspace{0.1in}\large{\textit{\hmwkClassInstructor\ \hmwkClassTime}}
    \vspace{3in}
}

\author{\textbf{\hmwkAuthorName}}
\date{}

\begin{document}

\maketitle

\pagebreak

\begin{homeworkProblem}
    Let \(A \leq_{L} B\) mean that \(A \leq_{T} B\) with additional condition
    that the oracle Turing machine \(M^B\) that solves \(A\) queries the oracle
    for \(B\) only once, at the very last step.
    \\

    Prove that \(A \leq_{L} B\) if and only if \(A \leq_{m} B\).

    \begin{proof}
        To prove this, we will show that each side implies the other.
        \\

        \textbf{Part One} If \(A \leq_{L} B\), then \(A \leq_{m} B\).
        \\

        Assume \(A \leq_L B\). This means there is an oracle TM \(M^B\) that
        can be queried once and only at the end execution.
        \\

        Let's construct a TM \(N\) that decides \(A \leq_m B\).
        \\

        \(N = \)`` On input \(w\):
        \begin{enumerate}
            \item Perform some computation using \(w\)
            \item Simulate \(M^B\) on newly computed value''
        \end{enumerate}

        This equivalent to a many-one reduction using oracle TM \(M^B\) because
        invoking \(M^B\) acts just as a computed function that maps \(A\) to
        \(B\). Thus it is simply a computed function.
        \\

        \textbf{Part Two} If \(A \leq_{m} B\), then \(A \leq_{L} B\).
        \\

        Assume that \(A \leq_m B\). This means there is a computable function
        \(f\) that reduces \(A\) to \(B\). Also this means there are two deciders, \(M_1\)
        and \(M_2\) that decide the languages for \(A\) and \(B\) respectively.
        \\

        To show what we'd like to prove, we will construct a new TM \(M\) using
        this function that operates exactly like a one-time use oracle executed
        last.
        \\

        Let's construct a TM \(N\) that decides \(A \leq_L B\).
        \\

        \(N = \)`` On input \(w\):
        \begin{enumerate}
            \item Perform some computation on \(w\) using \(M_1\)
            \item Compute \(f(w)\)
            \item Simulate \(M_2\) on newly computed value''
        \end{enumerate}

        Since we have proven each side, we have shown that \(A \leq_{L} B\) if
        and only if \(A \leq_{m} B\). Thus our proof is complete.

    \end{proof}
\end{homeworkProblem}

\pagebreak

\begin{homeworkProblem}
    Describe two different Turing machines, \(M_1\) and \(M_2\) such that when
    started on any input, \(M_1\) outputs \(\langle M_2 \rangle\) and \(M_2\)
    outputs \(\langle M_1 \rangle\).
    \\

    The two TMs are quite similar to the SELF program in that the first TM will
    print the encoding of the second machine and leave it on the tape. The second TM will
    then use that to compute what the first one is. We can define the two TM's like so:
    \\

    \(M_1 = P_{\langle M_2 \rangle}\)
    \\

    \(M_2 = \)`` On input \(\langle M \rangle\) where M is a TM:
    \begin{enumerate}
        \item Compute \(q(\langle M \rangle)\).
        \item Print newly computed TM and halt.''
    \end{enumerate}

    Similar to how Sipser explains the behavior for \(SELF\), we'll explain the
    behavior for this construction.

    \begin{enumerate}
        \item First \(M_1\) runs. It prints \(\langle M_2 \rangle\) on the tape.
        \item \(M_2\) starts. It looks at the tape and finds its own input, \(\langle M_2 \rangle\).
        \item \(M_2\) uses the lemma 6.1 in the book to calculate \(q(\langle M_2 \rangle)\) which is equal to \(\langle M_1 \rangle\).
        \item \(M_2\) then prints this newly computed description and halts.
    \end{enumerate}

    \textbf{Note:} This whole machine has the description of \(M_1 M_2\). This
    works because a program is just a string and thus two can be concatenated.
\end{homeworkProblem}

\pagebreak

\begin{homeworkProblem}
    Prove that no universal corruptor exists.

    \begin{proof}
        We will show that no universal corruptor exists by proof by contradiction.
        \\

        Assume that a universal corruptor does exist. Let this corruptor be the
        function \(f\). This function when given any TM, it will construct a TM
        that behaves differently. Formally this means \(L(M) \not= L(f(M))\).
        \\

        Given that this function exists, there should be no such TM that violates this
        corruptability.
        \\

        Let's create a new TM, \(C\) that tries to violate this:
        \\

        \(C = \)`` On input w:
        \begin{enumerate}
            \item Obtain using the recursion theorem, our own description, \(\langle C \rangle\).
            \item Compute \(f(\langle C \rangle)\) to obtain a new description of a TM, \(UNTOUCHABLE\).
            \item Simulate \(UNTOUCHABLE\) on \(w\).''
        \end{enumerate}

        Since \(f\) is universal, there cannot exist a TM that has the same
        language as the output of \(f\).
        \\

        Knowing this we can see that this is a contradiction for any universal
        corruptor \(f\). If \(f\) is in fact universal, then there is no TM
        that has the same language as the output of \(f\). Yet \(C\) and the
        new TM \(UNTOUCHABLE\) have the same language because \(C\) simulates
        \(UNTOUCHABLE\). This means that \(UNTOUCHABLE\) is in fact
        uncorruptable and thus untouchable.

    \end{proof}

    This proof follows the same idea that is presented in \textbf{Theorem 6.8}.
    Which shows that for any such transforamtion of a TM description, there
    exists some TM whose behavior is unchanged by the transformation.

\end{homeworkProblem}

\pagebreak

\begin{homeworkProblem}
    Let \(SELF_{TM} = \{ \langle M \rangle : L(M) = \{\langle M \rangle \}\}\).
    Prove that neither \(SELF_{TM}\) nor \(\overline{SELF_{TM}}\) is
    Turing-recognizable.
    \\

    We will use the same idea that is presented in \textbf{Theorem 6.5} by
    using recursion theorem to prove the unrecognizability of a language with a
    TM.
    \\

    \textbf{Part One} Prove that \(SELF_{TM}\) is not Turing-recognizable.

    \begin{proof}
        We will show that \(SELF_{TM}\) is not Turing-recognizable by proof by contradiction.
        \\

        Assume that \(SELF_{TM}\) is Turing-recognizable. Then there is a TM,
        \(M\) that recognizes \(SELF_{TM}\).
        \\

        We will construct the following TM, \(N\) to obtain a contradiction:
        \\

        \(N = \)`` On input w:
        \begin{enumerate}
            \item Obtain using the recursion theorem, our own description, \(\langle N \rangle\).
            \item Run \(M\) on input \(\langle N \rangle\).
            \item If \(M\) \(rejects\), then \(accept\), else \(reject\).''
        \end{enumerate}

        \textbf{TODO:} Give explanation of why it is a contradiction.
    \end{proof}

    \textbf{Part Two} Prove that \(\overline{SELF_{TM}}\) is not Turing-recognizable.
    \\

    \begin{proof}
    \textbf{TODO:} Meet with Professor Zhang


    \end{proof}
\end{homeworkProblem}

\pagebreak

\begin{homeworkProblem}
    Prove that the class \(P\) is closed under union and complementation.
    \\

    \textbf{Part One} Prove that \(P\) is closed under union.

    \begin{proof}
        To prove that \(P\) is closed under union, we assume that there are two languages
        \(L_1\) and \(L_2\) with \(M_1\) and \(M_2\) as TMs that decide them.
        \\

        Assume that \(M_1\) runs in polynominal time, \(O(n^x)\) as well as
        \(M_2\), \(O(n^y)\).
        \\

        We will construct a new TM \(M\) that runs both \(M_1\) and \(M_2\) and show that it still
        runs in polynomial time.
        \\

        \(M = \)`` On input \(w\):
        \begin{enumerate}
            \item Run \(M_1\) on \(w\).
            \item Run \(M_2\) on \(w\).
            \item If one of the TMs accepted, accept, else reject.''
        \end{enumerate}

        The runtime of \(M\) can be determined as \(O(n^x) + O(n^y)\). Asymptotically,
        this is equal to the following: \(O(n^{z})\) where \(z = max(x, y)\).
        \\

        Thus we can see that P is closed under union.
    \end{proof}

    \textbf{Part Two} Prove that \(P\) is closed under complementation.

    \begin{proof}
        To prove that \(P\) is closed under complementation, we assume that there is a
        language \(L\) with a TM \(M\) that decides it.
        \\

        Assume that \(M\) runs in polynominal time, \(O(n^x)\).
        \\

        We will construct a new TM \(N\) that runs \(M\) and we will
        show that it still runs in polynomial time.
        \\

        \(N = \)`` On input \(w\):
        \begin{enumerate}
            \item Run \(M\) on \(w\).
            \item If \(M\) accepted, reject, else accept.''
        \end{enumerate}

        The construction of \(N\) is such that it decides the complement of the
        language that \(M\) decides. This TM also runs in \(O(n^x)\), which
        means it still runs in polynomial time.
        \\

        Thus we can see that P is closed under complementation.

    \end{proof}
\end{homeworkProblem}

\end{document}
