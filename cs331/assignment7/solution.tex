\documentclass{article}

\usepackage{fancyhdr}
\usepackage{lastpage}
\usepackage{extramarks}
\usepackage[usenames,dvipsnames]{color}
\usepackage{amsmath}
\usepackage{amsthm}
\usepackage{amsfonts}

\usepackage{tikz}

\usetikzlibrary{automata,positioning,calc}

\topmargin=-0.45in
\evensidemargin=0in
\oddsidemargin=0in
\textwidth=6.5in
\textheight=9.0in
\headsep=0.25in

\linespread{1.1}

\pagestyle{fancy}
\lhead{\hmwkAuthorName}
\chead{\hmwkClass\ (\hmwkClassInstructor\ \hmwkClassTime): \hmwkTitle}
\rhead{\firstxmark}
\lfoot{\lastxmark}
\cfoot{}
\renewcommand\headrulewidth{0.4pt}
\renewcommand\footrulewidth{0.4pt}

\setlength\parindent{0pt}

\newcommand{\enterProblemHeader}[1]{
    \nobreak\extramarks{#1}{#1 continued on next page\ldots}\nobreak
    \nobreak\extramarks{#1 (continued)}{#1 continued on next page\ldots}\nobreak
}

\newcommand{\exitProblemHeader}[1]{
    \nobreak\extramarks{#1 (continued)}{#1 continued on next page\ldots}\nobreak
    \nobreak\extramarks{#1}{}\nobreak
}

\setcounter{secnumdepth}{0}
\newcounter{homeworkProblemCounter}

\newcommand{\homeworkProblemName}{}
\newenvironment{homeworkProblem}[1][Problem \arabic{homeworkProblemCounter}]{
    \stepcounter{homeworkProblemCounter}
    \renewcommand{\homeworkProblemName}{#1}
    \section{\homeworkProblemName}
    \enterProblemHeader{\homeworkProblemName}
}{
    \exitProblemHeader{\homeworkProblemName}
}

\newcommand{\problemAnswer}[1]{
\noindent\framebox[\columnwidth][c]{\begin{minipage}{0.98\columnwidth}#1\end{minipage}}
}

\newcommand{\homeworkSectionName}{}
\newenvironment{homeworkSection}[1]{
    \renewcommand{\homeworkSectionName}{#1}
    \subsection{\homeworkSectionName}
    \enterProblemHeader{\homeworkProblemName\ [\homeworkSectionName]}
}{
    \enterProblemHeader{\homeworkProblemName}
}

\newcommand{\hmwkTitle}{Homework\ \#7}
\newcommand{\hmwkDueDate}{April 1, 2013 at 11:59pm}
\newcommand{\hmwkClass}{CS331}
\newcommand{\hmwkClassTime}{9:00am}
\newcommand{\hmwkClassInstructor}{Professor Zhang}
\newcommand{\hmwkAuthorName}{Josh Davis}

\title{
    \vspace{2in}
    \textmd{\textbf{\hmwkClass:\ \hmwkTitle}}\\
    \normalsize\vspace{0.1in}\small{Due\ on\ \hmwkDueDate}\\
    \vspace{0.1in}\large{\textit{\hmwkClassInstructor\ \hmwkClassTime}}
    \vspace{3in}
}

\author{\textbf{\hmwkAuthorName}}
\date{}

\begin{document}

\maketitle

\pagebreak

\begin{homeworkProblem}
    Let \(K = \{\left<M\right> : \left<M\right> \notin L(M)\}\). Prove that \(K\) is not Turing
    recognizable.

    \begin{proof}
        To prove that \(K\) is not Turing recognizable, we will prove this by coming up with a contradiction.
        \\

        Assume that \(K\) is Turing recognizable and \(N\) is a Turing machine
        that recognizes it. \(K\) is the set of all encoded Turing machines
        that recognize the language that doesn't include their own encoding.
        \\

        Using this assumption, we can see that if we take the TM that
        recognizes \(K\), \(N\), we can see there are two cases in which this
        property of \(K\) still holds.
        \\

        \textbf{Case One}
        \\

        If \(N \in K\) then \(N \notin K\). This is because if it contains
        itself, then it must be removed because \(K\) cannot contain the
        encoding of itself.
        \\

        \textbf{Case Two}
        \\

        If \(N \notin K\) then \(N \in K\). This is because the encoding of
        \(N\) must be represented in \(K\) because it is the set of all TM that
        don't recognize themselves.
        \\

        Thus \(N \in K \iff N \notin K\).  This is clearly a contradiction and
        thus we have shown that \(K\) is not Turing recognizable.
    \end{proof}

\end{homeworkProblem}

\end{document}
